
\section{Введение}
        
    \par B-деревья были изобретены Рудольфом Байером и Эдвардом М. МакКрейтом во время работы в Boeing Research Labs(американская корпорация)  с целью эффективного управления индексными страницами для больших файлов с произвольным доступом. Основное предположение заключалось в том, что индексы будут настолько объемными, что в основную память смогут поместиться только небольшие фрагменты дерева. Статья Байера и МакКрейта «Организация и обслуживание больших упорядоченных индексов» была впервые опубликована в июле 1970 года.
    \par У В-дерева есть версии – это B+-tree и B*-tree.
    \par B+-дерево, в котором все значения сохранялись в листовых узлах, систематически рассмотрен в 1979 году.
    \par B*-деревья предложили Рудольф Байер и Эдвард МакКрейт, изучавшие проблему компактности B-деревьев. B*-дерево относительно компактнее, так как каждый узел используется полнее. В остальном же этот вид деревьев не отличается от простого B-дерева. 
    \par В деревьях поиска, таких как двоичное дерево поиска, AVL дерево, красно-черное и др., каждый узел содержит только одно значение и максимум двух потомков. В чем же проблема этих деревьев? К примеру, существует огромная база данных, представленная в виде одного из вышеперечисленных деревьев. Очевидно, что невозможно хранить все это дерево в оперативной памяти, следовательно в ней храниться лишь часть информации, остальное же храниться на стороннем носителе, допустим на жестком диске, скорость доступа к которому гораздо медленнее. В таком случае, красно-черное или декартово дерево будут требовать \(log(n)\) обращений к стороннему носителю. При больших n это слишком много. Как раз эту проблему и призваны решить B-деревья! 
    \par В-деревья могут хранить множество ключей в одном узле и могут иметь несколько дочерних узлов, что значительно уменьшает высоту, обеспечивая быстрый доступ к диску.
    \newpage
    
